% ===============================================
% frontmatter/eidesstattliche_erklaerung.tex - Eidesstattliche Erklärung
% ===============================================

% Signaturlinien mit Datum über der linken Linie
\newcommand{\signatureline}[2]{%
    \begin{minipage}[t]{0.4\textwidth}
        \vspace{0.8cm}
        \centering
        #1 % Datum über der linken Linie
        \vspace{0.1cm}\\
        \rule{6cm}{0.4pt}\\
        \vspace{0.05cm}
        Ort, Datum % Beschriftung unter der Linie
    \end{minipage}
    \hfill
    \begin{minipage}[t]{0.4\textwidth}
        \vspace{0.8cm}
        \centering
        \phantom{#1} % Leer über der rechten Linie  
        \vspace{0.1cm}\\
        \rule{6cm}{0.4pt}\\
        \vspace{0.05cm}
        #2 % Name unter der Linie
    \end{minipage}
}

\chapter*{Eidesstattliche Erklärung}
\addcontentsline{toc}{chapter}{Eidesstattliche Erklärung}

Ich erkläre an Eides statt, dass ich die vorliegende Arbeit selbstständig verfasst, andere als die angegebenen Quellen/Hilfsmittel nicht benutzt und die den benutzten Quellen wörtlich und inhaltlich entnommenen Stellen als solche kenntlich gemacht habe. Für die Erstellung der Arbeit habe ich auch folgende Hilfsmittel generativer KI-Tools [z.\,B. ChatGPT, Grammarly Go, Midjourney] zu folgendem Zweck verwendet:
\vspace{1.5cm}

% Signaturfelder für handschriftliche Unterschriften
% Verwendung: \signatureline{Datum}{Name}
\signatureline{Wien, xx.xx.xxxx}{Schueler 1}

\vspace{0.8cm}

\signatureline{Wien, xx.xx.xxxx}{Schueler 2}

\vspace{0.8cm}

\signatureline{Wien, xx.xx.xxxx}{Schueler 3}

\vspace{0.8cm}

\signatureline{Wien, xx.xx.xxxx}{Schueler 4}