% ===============================================
% main.tex - Hauptdatei des Templates
% ===============================================

\documentclass[12pt,a4paper,oneside]{report}

% --- Pakete und Einstellungen ---
\usepackage[utf8]{inputenc}               % UTF-8 Eingabekodierung
\usepackage[T1]{fontenc}                  % Zeichenkodierung
\usepackage[ngerman]{babel}               % Deutsche Silbentrennung und Übersetzungen
\usepackage{helvet}                      % Verbesserte Schriftart
\usepackage{setspace}                     % Zeilenabstand
\usepackage[left=3cm,right=2cm,top=2.5cm,bottom=2.5cm]{geometry} % Seitenränder
\usepackage{graphicx}                     % Grafiken einbinden
\usepackage{fancyhdr}                     % Kopf- und Fußzeilen
\usepackage{tocbibind}                    % Verzeichnisse im Inhaltsverzeichnis anzeigen
\usepackage{caption}                      % Verbesserte Beschriftungen
\usepackage{float}                        % Platzierung von Abbildungen und Tabellen
\usepackage{hyperref}                     % Hyperlinks und PDF-Metadaten
\usepackage{booktabs}                     % Hochwertige Tabellen
\usepackage{listings}                     % Quellcode-Listings
\usepackage{xcolor}                       % Erweiterte Farbunterstützung
\usepackage{glossaries}                   % Für Glossar und Abkürzungen
\usepackage{csquotes}                     % Für Zitate
\usepackage{titlesec}                     
\usepackage[style=numeric,backend=biber]{biblatex} % Literaturverzeichnis

% --- Pfad zur Bibliografie-Datei ---
\addbibresource{backmatter/literatur.bib}

% --- Glossar initialisieren ---
\makeglossaries

% --- Kopf- und Fußzeilen ---
\pagestyle{fancy}
\fancyhf{}                                % Alle Kopf- und Fußzeilenfelder leeren
\fancyhead[LE,RO]{\thepage}               % Seitenzahl oben außen
\fancyhead[RE]{\nouppercase{\leftmark}}   % Kapitel oben rechts auf geraden Seiten
\fancyhead[LO]{\nouppercase{\rightmark}}  % Abschnitt oben links auf ungeraden Seiten
\renewcommand{\headrulewidth}{0.4pt}      % Dicke der Trennlinie in der Kopfzeile

% --- Einstellungen für Code-Listings ---
\lstset{
  basicstyle=\small\ttfamily,
  keywordstyle=\color{blue}\bfseries,
  commentstyle=\color{green!60!black},
  stringstyle=\color{red},
  numbers=left,
  numberstyle=\tiny,
  numbersep=5pt,
  frame=single,
  breaklines=true,
  showstringspaces=false
}

% --- PDF-Metadaten ---
\hypersetup{
  pdftitle={Titel},
  pdfauthor={Schueler1},
  pdfsubject={Diplomarbeit},
  colorlinks=true,
  linkcolor=blue,
  citecolor=green!50!black,
  urlcolor=magenta
}

% --- Einrückung der Absätze deaktivieren ---
\setlength{\parindent}{0pt}
\setlength{\parskip}{0.5em}

% --- Zeilenabstand ---
\onehalfspacing

\begin{document}

% --- Kapitel Text unterdrücken ---
\titleformat{\chapter}[display]
  {\normalfont\huge\bfseries}  % Format des Titels
  {}                           % Leer - entfernt "Kapitel X"
  {0pt}                        % Abstand zwischen Label und Titel
  {\huge}                      % Vor dem Titel
  []                           % Optional nach dem Titel

% --- Abstand vor Kapitel weg machen ---
\titlespacing*{\chapter}{0pt}{0pt}{10pt}

% ===============================================
% Vorspann (Frontmatter)
% ===============================================

\pagenumbering{arabic}  % Arabische Zahlen für alle Seiten
% --- Titelseite ---
% ===============================================
% frontmatter/titlepage.tex - Titelseite der Dokumentation
% ===============================================

\begin{titlepage}

    % --- Kopfzeile mit Logos und Schule ---
    \noindent
    \begin{minipage}[t]{0.3\textwidth}
    \includegraphics[height=1.6cm]{bilder/schullogo.png}
    \end{minipage}
    \begin{minipage}[t]{0.4\textwidth}
    \centering
    \vspace{-1.5cm} % <-- Text etwas nach oben schieben
    \textbf{HTBLVA Wien 20}\\
    Höhere Lehranstalt für\\
    Biomedizin- und Gesundheitstechnik
    \end{minipage}
    \begin{minipage}[t]{0.3\textwidth}
    \raggedleft
    \includegraphics[height=1.6cm]{bilder/HTL_Logo.png}
    \end{minipage}


    \vspace{2cm}

    % --- Art der Arbeit ---
    \begin{center}
        {\large\textsc{Diplomarbeit}\par}
        \vspace{1.5cm}
        
        % --- Titel und Untertitel ---
        {\LARGE\textbf{Titel}\par}
        \vspace{1cm}
        {\large Untertitel \par}
        \vspace{2.5cm}
        
        % --- Autoren ---
        \begin{tabular}{ll}
            \textbf{Verfasst von:} & Schueler1, 5xHBG - Projektleiter \\
            & Schueler 2, 5xHBG \\
            & Schueler 3, 5xHBG \\
            & Schueler 4, 5xHBG \\
        \end{tabular}
        \vspace{1cm}
        
        % --- Betreuer ---
        \begin{tabular}{ll}
            \textbf{Betreut von:} & Lehrer\\
        \end{tabular}
        \vspace{1.5cm}
        
        ausgeführt im Schuljahr xx/xx
        
        \vspace{1.0cm}

    \end{center}

    % --- Fußzeile mit Abgabevermerk ---
    \vfill
\noindent
\begin{minipage}[t]{0.5\textwidth}   % <-- [t] sorgt für top alignment
    \textbf{Abgabevermerk:} \\
    Datum: dd.mm.yyyy
\end{minipage}
\begin{minipage}[t]{0.5\textwidth}   % <-- auch hier [t]
    \raggedleft
    \textbf{} \\
    übernommen von: \\
\end{minipage}


\end{titlepage}


% --- Eidelsstattliche Erklärung ---
% ===============================================
% frontmatter/eidesstattliche_erklaerung.tex - Eidesstattliche Erklärung
% ===============================================

% Signaturlinien mit Datum über der linken Linie
\newcommand{\signatureline}[2]{%
    \begin{minipage}[t]{0.4\textwidth}
        \vspace{0.8cm}
        \centering
        #1 % Datum über der linken Linie
        \vspace{0.1cm}\\
        \rule{6cm}{0.4pt}\\
        \vspace{0.05cm}
        Ort, Datum % Beschriftung unter der Linie
    \end{minipage}
    \hfill
    \begin{minipage}[t]{0.4\textwidth}
        \vspace{0.8cm}
        \centering
        \phantom{#1} % Leer über der rechten Linie  
        \vspace{0.1cm}\\
        \rule{6cm}{0.4pt}\\
        \vspace{0.05cm}
        #2 % Name unter der Linie
    \end{minipage}
}

\chapter*{Eidesstattliche Erklärung}
\addcontentsline{toc}{chapter}{Eidesstattliche Erklärung}

Ich erkläre an Eides statt, dass ich die vorliegende Arbeit selbstständig verfasst, andere als die angegebenen Quellen/Hilfsmittel nicht benutzt und die den benutzten Quellen wörtlich und inhaltlich entnommenen Stellen als solche kenntlich gemacht habe. Für die Erstellung der Arbeit habe ich auch folgende Hilfsmittel generativer KI-Tools [z.\,B. ChatGPT, Grammarly Go, Midjourney] zu folgendem Zweck verwendet:
\vspace{1.5cm}

% Signaturfelder für handschriftliche Unterschriften
% Verwendung: \signatureline{Datum}{Name}
\signatureline{Wien, xx.xx.xxxx}{Schueler 1}

\vspace{0.8cm}

\signatureline{Wien, xx.xx.xxxx}{Schueler 2}

\vspace{0.8cm}

\signatureline{Wien, xx.xx.xxxx}{Schueler 3}

\vspace{0.8cm}

\signatureline{Wien, xx.xx.xxxx}{Schueler 4}

% --- Kurzfassung (deutsch) ---
% ===============================================
% frontmatter/kurzfassung_de.tex - Deutsche Kurzfassung
% ===============================================

\chapter*{Kurzfassung}
\addcontentsline{toc}{chapter}{Kurzfassung}




% --- Kurzfassung (Englisch) ---
% ===============================================
% frontmatter/abstract.tex - Kurzfassung Englisch
% ===============================================

\chapter*{Abstract}
\addcontentsline{toc}{chapter}{Abstract}


% --- Inhaltsverzeichnis ---
\tableofcontents
\clearpage

% --- Einleitung ---
% ===============================================
% frontmatter/Einleitung.tex - Einleitung
% ===============================================

\chapter*{Einleitung}
\addcontentsline{toc}{chapter}{Einleitung}

% ===============================================
% Hauptteil (Mainmatter)
% ===============================================






% ===============================================
% Schlussteil (Backmatter)
% ===============================================

% --- Zusammenfassung ---
% ===============================================
% backmatter/zusammenfassung.tex - Zusammenfassung
% ===============================================

\chapter*{Zusammenfassung}
\addcontentsline{toc}{chapter}{Zusammenfassung}

% --- Abbildungsverzeichnis ---
\listoffigures
\clearpage

% --- Literaturverzeichnis ---
\nocite{*}
\printbibliography[heading=bibintoc]

\end{document}